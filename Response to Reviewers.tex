\documentclass[10pt,letterpaper]{article} % {{{
\usepackage[top=0.85in,left=2.75in,footskip=0.75in]{geometry}
\usepackage{amsmath,amssymb}
% Use adjustwidth environment to exceed column width (see example table in text)
\usepackage{changepage}

% textcomp cpackage and marvosym package for additional characters
\usepackage{textcomp,marvosym}
\usepackage{subcaption}


\usepackage[T1]{fontenc}
\usepackage{lmodern}
% cite package, to clean up citations in the main text. Do not remove.
\usepackage{cite}

% Use nameref to cite supporting information files (see Supporting Information section for more info)
\usepackage{nameref,hyperref}

% line numbers
\usepackage[right]{lineno}

% ligatures disabled
\usepackage[nopatch=eqnum]{microtype}
\DisableLigatures[f]{encoding = *, family = * }

% color can be used to apply background shading to table cells only
\usepackage[table]{xcolor}

% array package and thick rules for tables
\usepackage{array}
\usepackage{caption}
\captionsetup[figure]{labelfont={bf},labelformat={default},labelsep=period,name={Fig}}


% create "+" rule type for thick vertical lines
\newcolumntype{+}{!{\vrule width 2pt}}

% create \thickcline for thick horizontal lines of variable length
\newlength\savedwidth
\newcommand\thickcline[1]{%
  \noalign{\global\savedwidth\arrayrulewidth\global\arrayrulewidth 2pt}%
  \cline{#1}%
  \noalign{\vskip\arrayrulewidth}%
  \noalign{\global\arrayrulewidth\savedwidth}%
}

% \thickhline command for thick horizontal lines that span the table
\newcommand\thickhline{\noalign{\global\savedwidth\arrayrulewidth\global\arrayrulewidth 2pt}%
\hline
\noalign{\global\arrayrulewidth\savedwidth}}


% Remove comment for double spacing
%\usepackage{setspace} 
%\doublespacing

% Text layout
\raggedright
\setlength{\parindent}{0.5cm}
\textwidth 5.25in 
\textheight 8.75in

% Bold the 'Figure #' in the caption and separate it from the title/caption with a period
% Captions will be left justified
\usepackage[aboveskip=1pt,labelfont=bf,labelsep=period,justification=raggedright,singlelinecheck=off]{caption}
\renewcommand{\figurename}{Fig}
\renewcommand{\succ}{\textrm{succ}}

\newcommand{\fref}[1]{Fig~\ref{#1}}
\newcommand{\eref}[1]{Eq~(\ref{#1})}
% Use the PLoS provided BiBTeX style
\bibliographystyle{plos2015}

% Remove brackets from numbering in List of References
\makeatletter
\renewcommand{\@biblabel}[1]{\quad#1.}
\makeatother



% Header and Footer with logo
\usepackage{lastpage,fancyhdr,graphicx}
\usepackage{epstopdf}



\usepackage[english]{babel}
\usepackage{graphicx,helvet}
\usepackage{color}
\usepackage{url}
\usepackage{amssymb}
\usepackage[utf8]{inputenc}
\usepackage{hyperref}
% \usepackage[inline]{showlabels}
\usepackage{bbm,bm}
\usepackage{soul}
\usepackage{amsfonts}

\usepackage{tikz}
\usetikzlibrary{arrows}
\usetikzlibrary{quantikz}

\usepackage[draft,inline,nomargin]{fixme} \fxsetup{theme=color}
\FXRegisterAuthor{cp}{acp}{\color{blue}CP}
\FXRegisterAuthor{tb}{ttb}{\color{green}TB}




%\pagestyle{myheadings}
\pagestyle{fancy}
\fancyhf{}
%\setlength{\headheight}{27.023pt}
\rfoot{\thepage/\pageref{LastPage}}
\renewcommand{\headrulewidth}{0pt}
\renewcommand{\footrule}{\hrule height 2pt \vspace{2mm}}
\fancyheadoffset[L]{2.25in}
\fancyfootoffset[L]{2.25in}
\lfoot{\today}

%% Include all macros below

\newcommand{\lorem}{{\bf LOREM}}
\newcommand{\ipsum}{{\bf IPSUM}}
\DeclareMathOperator{\tr}{tr}
\newtheorem{definition}{Definition}
\newtheorem{theorem}{Theorem}

%% END MACROS SECTION

% }}}
\begin{document}
\section*{Response to Editor:} % {{{

We are re-submitting our paper ``Quantum simulation of Pauli channels and dynamical maps: algorithm and implementation''. 
We believe that the revised version of our paper addresses all
concerns by the referees in detail and is suitable for publication. 
% }}}
\section*{Reviewers' General Comments} % {{{
\textbf{Reviewer 1:} \\% {{{

\textbf{
The paper presents two results: 1. A way to simulate
parametric Pauli Channels using a gate-based formalism, useful for
quantum computation, and 2. An investigation on the class of
parametric Pauli Channels that can be parametrised by acting only on
one qubit via one parameter.
The full detailed comments on the paper are attached in the pdf file.
My general comments are the following:
\begin{itemize}
\item[1.]The two aforementioned results are disjointed and completely
lacking in motivation. The entire logic of the paper is obscure, and
seemingly lacks any ground for the results to be useful in any way.
\item[2.] The simulation on the IBM quantum computer is irrelevant, and does
not prove neither the validity of the decomposition (as it fails for
most of the range of operations) nor it is quantifiying any relevant
parameter (noise for example). Data from those experimental runs and
complete methodology for calculating the diamond norm is also missing.
\item[3.] The claims are vague, and the mathematical formalism is vague, with
strong abuse of notation and non-strict definitions and theorems.
There are also many spelling mistakes and refuses, even in the
formulas.
\item[4.] The analysis on the 1-parametrisation is lacking a discussion on
what happens at higher than 2 dimensions.
\end{itemize}
For the aformenetioned reasons, I cannot reccomend this paper for
publication. I suggest the following three course of actions:
\begin{itemize}
\item[1.] Separate the two main results.
\item[2.]Improve the tightness of the mathematical formalism.
\item[3.]  Find an application of the generalised Pauli Channels that can be
effectively simulated with the methodology.
\end{itemize}
} 

We thank the first reviewer for the time spent carefully 
assessing the manuscript, and the opinions given.
We have made major changes in the manuscript following your comments and recommendations. 
Some of the general changes we made are explained in the following list
and then we respond to more specific comments
throughout the text.
\begin{itemize}
\item We rewrote big parts of the introduction to make clearer the logic and motivation of the paper. 
We begin the paper by presenting the theory of Pauli channels, 
and then show the first result, which is the algorithm for simulating them on a quantum computer,
which we try out for systems of one and two qubits.
With that algorithm in mind, we seek to generalize it to Pauli dynamical maps, 
so as to include continuous transformations of the 
N-qubit system. This goal leads us to
focus on parametrized quantum circuits and then to consider
what can be done with only moving one parameter.
\item We added the data and code from the IBM quantum computer simulations
to a Github repository: \href{https://github.com/tbasile/Quantum-simulation-of-Pauli-Channels.git}{https://github.com/tbasile/Quantum-simulation-of-Pauli-Channels.git}
\item We improved the tightness of the mathematical formulation by adding
a more mathematical definition of 1PR circuits and
rephrasing the proof of theorem 2 to make it clearer. 
The specific changes are discussed below as responses to the comments made by the reviewer throughout the text.
\item In section 5 of the paper we expanded the explanations on the examples for Pauli dynamical maps
that can be implemented with a 1PR circuit.
These Pauli dynamical maps are well-known and we prove here
that they can be done with only moving one parameter in the circuit. We also added explicitly the matrices $A$ and $B$ that can
be used to simulate them with a 1PR circuit.
Furthermore, we included Fig. 8, where
we show the results of implementing these dynamical circuits on IBM's computer using a 1PR circuit.

\end{itemize}



% }}}
\textbf{Reviewer 2:} \\ % {{{
\textbf{The authors introduced some quantum circuits to implement the action
of arbitrary Pauli channels using fixed-state ancilla qubits, and
certain single-parameter Pauli dynamical channels using
one-parameter-rotation circuits consisting of one parametrized
single-qubit rotation and two unparametrized multiqubit unitary
operations. They showed that the latter class of dynamical channels
implementable is related to the one-parameter ancilla-state curves
allowable for such circuits. This is understandable since only a
single-parameter controlled rotation is used for the circuit, which
does not offer sufficient flexibility to realize all single-parameter
Pauli dynamical channels from my understanding.}

\textbf{While the implementations appear to be mathematically sound, there
appears to be a lack of comparisons between the actual channel
implemented and the target channel. There is only one Fig. 3 that
shows the diamond fidelities for homogenously-chosen single-qubit
unparametrized Pauli target channels within the tetrahedron region.
This is a nice plot that reveals the realistic errors coming from the
IBM quantum cloud, namely that the single-qubit Pauli channels
corresponding to the central part, which I presume are the
depolarizing channels, have the highest diamond fidelity.}

\textbf{
Besides this plot, the authors however did not show the performance of
their algorithms for larger number of qubits, and there are no
performance plots for dynamical Pauli-channel implementations. For a
further evaluation as to whether the algorithms are scientifically
practical, at least a plot on two-qubit Pauli channels and one on
two-qubit dynamical Pauli channels. My take is that because the
unitary operators A and B are too sophisticated for hardware
implementations, since they are even more complicated than single
controlled unitary operators, the fidelity could be very strongly
peaked around the zero of the pseudo Bloch-vector space for multiqubit
systems. If the overall system is too large even for simulating
two-qubit channels (due to the ancillas), then classical numerical
simulations are sufficient.}

\textbf{
In any case, after supplying the performance plots, I believe that the
algorithms and performance plots shall supply a fair presentation on
the scientific soundness of this work.} \\$\;$\\


We thank the second reviewer for the detailed analysis
of our manuscript and the comments made. 
Following the recommendations, we 
added section 3.3 in which we simulate a two qubit Pauli dynamical map 
and in Fig. 4 show that the fidelities are still reasonably high. 

Regarding the 1PR circuits, we added explicitly the matrices $A$ and $B$
to the examples of dynamical maps and
simulated them on IBM's quantum computer
using the dynamical Pauli channel implementation.
For each of these simulations, we used the proposed 1PR circuit and plotted  in Fig. 8 the fidelities for
different channels along the dynamical map.

Finally, we included a paragraph in the conclusion discussing what happens on higher dimensions and arguing that paradigmatic
channels such as the two qubit dephasing channel of section 3.3 will still have a 
1PR circuit implementation. 


% }}}
% }}}
\section*{Comments throughout the text} % {{{

Here we present the comments made by the reviewers throughout the text 
and respond to each of them individually.
The comments are written in bold and our response in regular font.

\begin{enumerate}
\item \textbf{I fail to see the point here, the logic of the paper is: 
1. ``open quantum systems can be modeled through a class of channels representing he various types of decoherence.'' 2. ``Quantum circuits are a convenient gate-based formalism that is well understood and has an extensive body of work, and experimentally accessible.''
3. We map (generalised) Pauli channels onto quantum circuits, as a way to simulate them.} 

\textbf{What I am missing is what do we learn by doing it which cant be learned by actually performing the Pauli channel onto a quantum system? For example, applying well controlled sx, sy, sz operations and their coherent parametric combination using a photonic polarisation or path encoding is experimentally trivial. Why would I need to decompose (not even effectively for any channel) the pauli channels to match a gate based description? Which class of problems can we simulate that cannot be simulated otherwise? Is there any advantage (for example in speed up, resources like ancillas required, resistance to unwanted noise etc..?)}

\textbf{Intuitively, it seems incredibly harder resource wise to create the sum b|gamma> superposition state needed for the ancilla, just for the sake of having it into a "quantum circuit" perspective. }\\


We rewrote parts of the introduction to make the objective of the manuscript clearer. 
First of all, one of the principal objectives of quantum computers nowadays
is to use them for simulation of quantum processes,
and out of all quantum processes, decoherence is one of the most important and prevalent.
That is why we chose to simulate Pauli channels, which are one of the
simplest descriptions of decoherence. 
Important characteristics about the algorithm proposed for Pauli channels
are its simplicity and generality.
We chose to represent the channel using a gate-based formalism because
of it being one of the most common and general ways of
representing quantum algorithms.

Furthermore, the simplicity and generality of the circuit proposed allows us to
naturally generalize it to Pauli dynamical maps by just
adding movable parameters to the circuit.
Finally, we show the general result of what parametrized quantum operations can be done 
moving only one parameter, which experimentally could translate
to having to adapt only one setting of an experiment
to simulate the quantum operation.




\item \textbf{ Really confusing notation, I suggest use the much more well
known relation using the Levi-Civita tensor.}\\


A sentence was added to the text mentioning how we get to the equation by
starting with the relation $[\sigma_\alpha,\sigma_\beta] = 2 i
\epsilon_{\alpha \beta \gamma} \sigma_\gamma$.
After using this equation on $\sigma_{\alpha} \sigma_{\gamma} \sigma_{\alpha}$, the Levi-Civita tensors are summed out and we are left with:
\begin{equation*}
\sigma_{\gamma} \sigma_{\alpha} \sigma_{\gamma}  =   A_{\alpha,\gamma} \sigma_{\alpha}, \;\;\;\;\;\; \text{with} \; A_{\alpha, \gamma} = (2\delta_{\alpha \gamma}-1)(2\delta_{\alpha 0}-1)(2\delta_{\gamma 0}-1)
\end{equation*}
We believe that writing matrix $A$ explicitly is simpler than this expression. 

\item \textbf{Eq 2} \\

The correct equation is indeed equation 1, since we are referring to the fact that 
\begin{equation}
\label{eq: rho-transformada}
\mathcal{E}(\rho) = \dfrac{1}{2} \sum_{\alpha} \left(\sum_{\gamma} A_{\alpha, \gamma} k_{\gamma} \right) r_{\alpha} \sigma_{\alpha}.
\end{equation}
has the form of $\dfrac{1}{2} \sum_{\alpha} r_{\alpha} \sigma_{\alpha}$, only with the
$r_{\alpha}$ replaced by $\left( \sum_{\gamma} A_{\alpha,\gamma} k_{\gamma} \right) r_{\alpha}$.


\item \textbf{At this point I am missing a proper defiition of a "quantum circuit" consistent with the mathematical formalism. The authors start from the very basic regarding Pauli channels, but give for granted the quantum circuit formalism.} \\

We added a few lines at the beginning of section 3 defining the concept of quantum circuit.
We  mention that quantum circuits are a representation of computation on quantum systems, which applies a unitary operator $U$. 

\item  \textbf{So another way to see this is that the simulation through the proposed decomposition works only for depolarising channels, and fails otherwise.}\\

The algorithm works to some extent for all Pauli channels, and how well
it does or does not work is quantified by the fidelity.
Indeed, the result we got is that it works much better for channels that are depolarizing
and less well as we get closer to unitary channels.


\item \textbf{Why?}\\
We rephrased the sentence to make the point clearer
and explained that as a consequence of the circuit being usable in general for any Pauli channel,
it implements unitary channels in a very roundabout way.


\item \textbf{But this is a severe disadvantage, that begs the question: what is the advantage?} \\

The advantages are that it is a straightforward, concrete and general way
of simulating Pauli channels.
Being general gives the possibility to later use it for Pauli dynamical maps by only needing to vary the parameters
and not the design of the circuit itself.


\item \textbf{This definition is vague - given the vagueness of the term "quantum circuit".
 Here what you are defining is just a unitary operation acting on a N-qubit Hilbert space, that is parameter dependent only locally on one specific qubit. } \\
 
We added a sentence explaining that to make it more concrete.

 
\item \textbf{Normal}\\

They are only orthogonal, since their individual norms are not equal to $1$. 
After restructuring the proof of theorem 2, this is now clearer.
 
\item \textbf{Again, this is not a conclusion from theorem 1. This is literally Definition 1. Also, use the proper notation here - make A and B operators living on an dim N Hilbert state, and R explixitly dependent on s and operating on qubit N. Then the matrix representation of the operators is a $2^n\times 2^n$ matrix which is applied to the state j. The explanation in words in not sufficient.} \\

We rephrased the first paragraph of the proof to make it easier to see when we use the result of theorem 1. 
Basically, we first use definition 1 to say that the circuit has the form $ARB$,
and then the result of theorem $1$
to conclude that $R$ is in particular a $\sigma_3$ rotation
of angle $2s$ applied to the last qubit.\\


\item \textbf{True only if the initial j states are normalised, but they were undefined.}\\

We added the definition of $|j\rangle$  at the end of the theorem statement.\\


\item \textbf{How can it make sense to define A and B applied respectively to the last 3 and the first qubit?}\\

We changed the phrasing to make it clearer that $A$ and $B$ 
are restricted to act in a particular way on the last 3 and first qubit respectively.
Besides from that, the remaining parts of the operators $A$ and $B$ can be chosen arbitrarily, 
with the only restriction that the resulting matrices are unitary. \\

\item  \textbf{Please, at least give an example of this semiarbitrary construction of A and B.}\\

We added examples of these matrices for the particular dynamical maps shown in section 5. \\

\item \textbf{I don't understand what operatively is happening here. Select the state 0, apply B, apply $R_s$, apply A and then "straight forward calculations" lead to what?} \\

We rephrased this part to make it clearer. The idea is that we are searching for the matrices $A$ and $B$ such
that for given vectors $|a\rangle, |b\rangle, |c\rangle$, the circuit creates
the state $e^{is(p)} |a \rangle + e^{-is(p)} |b \rangle + |c\rangle$.
In this part, we show that with the matrices $A$ and $B$ we constructed,
we indeed create said state.
That way, given any curve of states that can be done with a 1PR circuit
according to theorem 2, here we show how
to actually create said circuit (that is, which matrices $A$, $B$ to select).\\

\item \textbf{To see what?} \\

As said before, to see that the selection of matrices $A$, $B$ creates 
the curve of states we wanted.
This part of the text has been rewritten to make it clearer.\\


\item \textbf{So this is just the proof that the proposed operative definition of A and B is properly normalised, right?} \\

This is an intermediate step, but as said in the previous questions, 
we are proving that the selection of $A$ and $B$ for given vectors $|a\rangle, |b\rangle, |c\rangle$ 
indeed works to create the state
$e^{is(p)} |a \rangle + e^{-is(p)} |b \rangle + |c\rangle$.\\

\item \textbf{To be consistent with the previous formalism, the states should be 0,1,2,3}\\

We made the change.\\

\item \textbf{There is literally one parameter in one qubit, of course it depends on one parameter...
 how can this be a result?} \\
 
The result is that they needed  to move just one parameter to implement 
many Pauli channels of the bit flip map.
For other maps, they may need to move more than one parameter to 
implement all the channels in the map. 
However, just as we showed that the bit flip map can be done with a 1PR circuit,
they implemented the map using one parameter.\\
  
 \item  \textbf{This is interesting, instead of giving only some random examples, I would actually explore more in details the zoology of this class of dynamical maps - if lucky, you would find class of interesting problems that can be simulated effectively in an experimental setup by just accessing one parameter (as the authors say, just with one set of waveplates, for example)} \\
 
Some of the examples given (depolarizing, bit flip, phase flip, bit-phase flip) are very well known 
Pauli maps.
We also included random examples with the purpose of seeing how some general maps that 
can be implemented with a 1PR circuit look like.
 
 \item  \textbf{I feel that there needs to be a treatment of what happens, at least qualitatively at higher than two dimension, as one of the selling point is the generality of the formalism for N qubit. I expect, however, that the higher the dimension, the less rich is the space that can be parametrised by only one parameter, so.. what is the applicability of the theorem for higher dimensions?} \\
 
We added a paragraph in the conclusion talking about that.
There we argue that for bigger systems, the space of dynamical maps
that can be implemented with a 1PR circuit
is a smaller part of the whole space of dynamical maps.
However, some of the most paradigmatic maps will still be implementable this way.
  
\item \textbf{Imprecise} \\ 
 
We have rephrased parts of the conclusion to make it clearer.
 
 
\item \textbf{This reasoning I miss, simplifying for achieving what goal? } \\
 
We rephrased the conclusion to make it clearer. We wanted to reduce the number of parametrized gates
so as to have only one parameter that works as a kind of knob to move around in the channels forming
a dynamical map.
  
\item\textbf{ There is no quantification of this, it´s not even the focus of the paper.}  \\
 
We have a quantification of the fidelity of the implementation of different Pauli channel and dynamical maps.
 Even though it is indeed not the focus of the paper, we believe it
 is an important enough part of it as to mention it again in the conclusion.
 
\item \textbf{Citation required here.}
We added a citation.
\end{enumerate}
% }}}
\end{document}
