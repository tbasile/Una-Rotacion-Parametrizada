\documentclass[12pt, addrfooterall, english]{if_letter_2013}
% \usepackage[spanish]{babel} 
\usepackage[utf8]{inputenc}
\usepackage{hyperref}

% \date{17 de Octubre de 2018}
\begin{document}
\begin{letter}{
.\vspace{-2cm}
% PLOS One\\
% Editorial 
% \adirectorcecilia
% \aquieninterese
% \apapiitmendoza
% \aconacytGelover
% \asecretariomostalac
% Dra. Maria del Carmen de la Peza Casares\\
% Directora Adjunta de Desarrollo Científico y Secretaria Técnica del FONCICYT\\
% CONACYT
}
\opening{To whom it may concern}

% Para poder recibir apoyo económico del proyecto conacyt: Básicamente solo
% tienes que devolverme a mi estatus anterior, actualmente soy becario pero debo
% ser participante. Hay que escribir una carta dirigida a: 
% 
% 
% "
% Dra. Maria del Carmen de la Peza Cázares
% Directora adjunta de desarrollo científico
% y secretaria técnica del FONCICYT


% Deseo informarle que el día de hoy me reincorporé a mis actividades después de mi viaje
% a Colombia. 

% y que  posteriormente se le reintegren los gastos de un viaje que
% realizó recientemente (\$16,534.22 pesos) y del que ya se entregó la
% documentación. El motivo es que se deseaba otorgar una beca de doctorado al
% Maestro Dávalos (y se devolvió el dinero del viaje), pero no fue posible
% concretar la beca por problemas administrativos. 
% 
% La Licenciada Adela Lorán nos indico que conviene que le informemos por escrito a usted de esta situación. 

% Por medio de la presente me comprometo a no ausentarme del Instituto de Física por períodos
% superiores a un mes con el fin de cumplir con el compromiso de atender al Dr.
% Roberto Kenan Uriostegui, durante el tiempo que esté realizando su estancia posdoctoral 
% bajo mi dirección. 
% 
% 
% Sin más por el momento, me despido y agradezco su atención.

I am writing on behalf of all authors to submit our research article entitled
“Quantum simulation of Pauli channels and dynamical maps: algorithm and
implementation” to be considered for publication as a research article in PLOS
One. 

In this article, we focus on one of the fundamental purposes of quantum
computers since their inception: simulating quantum systems. Specifically, we
propose and simulate on a quantum computer an algorithm for implementing
Pauli channels. These channels are transformations of qubit systems typically 
arising from noise affecting quantum devices (Flammia S. and Wallman J.,
Efficient estimation of Pauli channels, 2020). Pauli channels are examples of 
the evolution of open quantum systems, which have garnered interest due to 
their applications in the study of entanglement (Farías O.J. et al, 
Observation of the emergence of multipartite entanglement between a bipartite 
system and its environment, 2012) and dissipative processes 
(Barreiro J. et al, An open-system quantum simulator with trapped ions, 2011).

Moreover, we extend the algorithm to encompass Pauli channels dependent on a
parameter, which we refer to as Pauli dynamical maps. This extension leads us
to investigate quantum algorithms with parameter-dependent operations, known as
parametrized quantum circuits.  In this manuscript, we establish the
mathematical conditions necessary for implementing any parametrized operation
using a quantum circuit using only one parametrized single qubit operation. We
believe these findings will be of interest to readers of your esteemed journal.

All relevant data and codes to replicate the results can be found in the following
public repository \href{https://github.com/tbasile/Quantum-simulation-of-Pauli-Channels.git}{https://github.com/tbasile/Quantum-simulation-of-Pauli-Channels.git}.

We declare the manuscript’s originality and exclusivity for publication. In
addition, we have not had any prior interactions with PLOS regarding this
manuscript. In the event that our manuscript is chosen for review, we kindly
suggest Dr. Fabio Sciarrino as a suitable editor, given his background and
expertise to evaluate our findings. We have no objections to any other
potential reviewers.

Each named author has significantly contributed to the research and preparation
of this manuscript. Furthermore, the named authors have no conflicts of
interest to disclose. Both authors received support from projects CONAHCyT 
(Consejo Nacional de Humanidades, Ciencias y Tecnologías
\href{https://dgapa.unam.mx/index.php/impulso-a-la-investigacion/papiit}{https://dgapa.unam.mx/index.php/impulso-a-la-investigacion/papiit}) 285754, 
and UNAM-PAPIIT  (Universidad Nacional Autónoma de México - Programa de Apoyo a 
Proyectos de Investigación e Innovación Tecnológica \href{https://conahcyt.mx/}{https://conahcyt.mx/}) IG101421.
The funders had no role in study design, data collection and analysis, 
decision to publish, or preparation of the manuscript.

\closing{On behalf of all authors,}

\end{letter}
\end{document}
recepcion_cb@conacyt.mx 
