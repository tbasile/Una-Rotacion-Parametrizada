\documentclass[varwidth, border=10pt]{standalone}
\usepackage{tikz}
\usepackage[]{xcolor}
\usepackage{subcaption}
% % \usetikzlibrary{positioning}
% \usetikzlibrary{arrows,backgrounds}
% 
% \usepackage{amsmath, amsthm, amssymb}
% \input{commondefinitions}
% 	
% 
% 
% \begin{document}
% \pagestyle{empty}
% %\providecommand{\openone}{\leavevmode\hbox{\small1\kern-3.8pt\normalsize1}}
% 
% % \documentclass[12pt]{article}
% % \usepackage{psfrag}
% \usepackage{tikz}
% % \usetikzlibrary{positioning}
% % \usetikzlibrary{arrows,backgrounds}
 \usetikzlibrary{calc}
\usepackage{bbold}

\usepackage{tikz}
\usetikzlibrary{arrows}
\usetikzlibrary{quantikz}


\usepackage{amsmath, amsthm, amssymb}
\colorlet{coscolor}{blue}

\newcommand{\red}{\color{red}}
\newcommand{\blue}{\color{blue}}
%Colors

\definecolor{max}{rgb}{1,0.54,0.1}
\definecolor{medium}{rgb}{1,0.8,0.6}
\definecolor{min}{rgb}{1,0.9,0.8}
\begin{document}
\thispagestyle{empty}

%\begin{figure} % {{{
%\centering
%\begin{subfigure}{0.9\textwidth}
%\begin{subfigure}{.666\textwidth}
%\centering
%\includegraphics[width=.95\columnwidth]{tetra-def.pdf}
%\end{subfigure}%
%\begin{subfigure}{.333\textwidth}
%\includegraphics[width=.95\columnwidth]{corte1-2.pdf} \\
%\includegraphics[width=.95\columnwidth]{corte2-2.pdf} \\
%\end{subfigure}
%\includegraphics[width=.322\columnwidth]{corte3-2.pdf}
%\includegraphics[width=.322\columnwidth]{corte4-2.pdf}
%\includegraphics[width=.322\columnwidth]{corte5-2.pdf}
%\end{subfigure}%
%\begin{subfigure}{0.1\textwidth}
%\includegraphics[width=.8\columnwidth]{col-2.pdf} \\ [1ex]
%\end{subfigure}
%%\caption{Results of the diamond fidelities as defined in \eref{eq:diamond-fid}
%%for a sample of Pauli channels in the tetrahedron.  Notice that channels close
%%to the center of the tetrahedron have high fidelities, while those close to the
%%borders do not.  Moreover, we show the results for cuts of the tetrahedron at
%%different values of $\tau_3$.}
%%\label{Fig3}
%\end{figure}
\newpage


\begin{tikzpicture}
\node[inner sep=0pt] (curvas) at (0,0)
	{\includegraphics[width=.60\textwidth]{tetra-def.pdf}};
\node[inner sep=0pt] (curvas-azar) at (5.5,1.5)
	{\includegraphics[width=.3\textwidth]{corte1-2.pdf}};
\node[inner sep=0pt] (curvas-azar) at (5.5,-2.2)
	{\includegraphics[width=.3\textwidth]{corte2-2.pdf}};
\node[inner sep=0pt] (curvas-azar) at (5.5,-5.9)
	{\includegraphics[width=.3\textwidth]{corte5-2.pdf}};
\node[inner sep=0pt] (curvas-azar) at (1.8,-5.9)
	{\includegraphics[width=.3\textwidth]{corte4-2.pdf}};
\node[inner sep=0pt] (curvas-azar) at (-1.9,-5.9)
	{\includegraphics[width=.3\textwidth]{corte3-2.pdf}};
\node[inner sep=0pt] (curvas-azar) at (8.0,-2)
	{\includegraphics[width=.08\textwidth]{col-2.pdf}};
\end{tikzpicture}

\end{document}



